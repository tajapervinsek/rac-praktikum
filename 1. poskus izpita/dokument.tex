\documentclass[a4paper,12pt]{article}
\usepackage[slovene]{babel}
\usepackage[utf8]{inputenc}
\usepackage[T1]{fontenc}
\usepackage{lmodern}
\usepackage{amsmath, amsthm, amssymb}

{\theoremstyle{definition}
\newtheorem{definicija}{Definicija}
}

\DeclareMathOperator{\Lin}{Lin}

\title{Nekaj o invariantnih podprostorih}
\author{Beno Učakar}
\date{}

\begin{document}

\maketitle

Naj bosta $V_1$ in $V_2$ vektorska prostora ter $L \colon V_1 \times V_2 \to V_1 \times V_2$ linearna preslikava na zunanji direktni vsoti prostorov $V_1$ in $V_2$.
Potem lahko na narave način definiramo linearni preslikavi $L_1 \colon V_1 \to V_1$ in $L_2 \colon V_2 \to L_V$, da velja  
\(L(x_1, x_2) = (Lx_1, Lx_2).\)
Kaj pa, če gledamo notranjo direktno vsoto? Tu bomo naleteli na pojem invariantnega podprostora.

    \begin{definicija}
    Vektorski podprostor $U$ je invarianten podprostor linearne preslikave $L$, če velja 
    $L(U) \subseteq U$
    \end{definicija}

Vidimo torej, da če je prostor $U$ invarianten, bo obnašanje preslikave $L$ nekako ostalo znotraj prostora $U$.
Na primer, če je $v \in V$ lastni vektor preslikave $L$ za lastno vrednost $\lambda$, je prostor 
\(U = \Lin \{ v \}\) invarianten podprostor linearne preslikave $L$.

% podnaslov
\section*{Reducirajoči podprostori}

Naj bo sedaj prostor $V$ notranja direktna vsota podprostorov $U_1$ in $U_2$, torej $V = U_1 \oplus U_2$.
Potem lahko vsak vektor $x \in V$ na enoličen način zapišemo kot $x = x_1 + x_2$, kjer je $x_1 \in U_1$ in $x_2 \in U_2$.
Če sta oba podprostora $U_1$ in $U_2$ še invariantna podprostora preslikave $L$, ju skupaj imenujemo reducirajoča podprostora.
Definirajmo linearni preslikavi 


% enačba v prikaznem načinu
\[L_1 = L \rvert_{U_1} \colon U_1 \to U_1 \quad \text{in} \quad L_2 = L \rvert_{U_2} \colon U_2 \to U_2.\]

Ker sta prostora $U_1$ in $U_2$ oba invariantna za preslikavo $L$, sta preslikavi $L_1$ in $L_2$ dobro definirani.
Za poljuben vektor $x \in V$ lahko potem zapišemo 


% enačba v prikaznem načinu
\begin{multline*}
Lx = L(x_1 + x_2) = Lx_1+ Lx_2 = L_1x_1 + L_2x_2 = (L_1 \oplus L_2)(x_1 + x_2) = \\
= (L_1 \oplus L_2)x,
\end{multline*}

torej velja $L = L_1 \oplus L_2$.
Če preslikavo $L$ zapišemo matrično glede na razcep $V = U_1 \oplus U_2$, vidimo, da je


% enačba v prikaznem načinu
\[
L = 
\begin{bmatrix}
 L_1 & 0 \\
 0 & L_2
\end{bmatrix}
\]

torej smo preslikavo $L$ bločno diagonalizirali.

Za nadaljnje branje priporočamo učbenik Sheldona Axlerja, Linear Algebra Done Right \cite{axler}.

\bibliographystyle{siam}
\bibliography{viri.bib}

\end{document}